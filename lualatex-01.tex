%! TeX program = lualatex

\documentclass{article}
\usepackage{luacode}
\usepackage[cwttf]{luatexja-zhfonts}
%%% 適合中文的行間距。
\renewcommand{\baselinestretch}{1.36}
%%% 消除中文問號及驚嘆號後面多餘的空白。
\def\exsp#1{\unskip#1\hspace{0em}}
\begin{luacode}
function dosub(s)
  s = string.gsub(s, '!', '\\exsp!')
  s = string.gsub(s, '?', '\\exsp?')
  return(s)
end
\end{luacode}
\AtBeginDocument{%
  \luaexec{luatexbase.add_to_callback("process_input_buffer", dosub, "dosub")}%
}
\begin{document}
在Lua\TeX{}中正常地使用中文。獲得自動的\textbf{字體選擇},標點「壓縮」,以及正確的斷行處理等特性。
\section{說文解字}
\textbf{《說文解字》}書名。\textit{東漢許慎}撰,三十卷,為我國第一部有系統分析字
及考究字源>的字書。按文字形體及偏旁構造分列五百四十部,首創部首編排法?字體以小篆為主,
收錄九千三百五十三字,列古文、籀文等異體為重文,共計一千一百六十三字。每字下的解釋大抵
先說字義,次及形體構造>及讀音,依據六書解說文字。晚近注家以清段玉裁、桂馥、朱駿聲、王
筠最為精博。簡稱為「說文」。

戰國策,書名!漢劉向輯,三三卷。依國別分為西周、東周、秦、齊、楚、趙、魏、韓、燕、宋、衛、
中山十二策。內容記載戰國時代各諸侯國的政治大事及當時策士的言論行動。描寫人事,運用誇飾、
比{\Uchar"55A9}、排比等手法,語言形象鮮明,辯辭宏肆瑰麗,刻劃歷史人物亦活躍生動,
為我國重要的史料。今通行有漢高誘注本。也稱為「短長書」、「國策」、「長書」。{\Uchar"55BB}
\section{漢字}
\textbf{漢字}用以記錄漢語的傳統文字Traditional character。形聲字占絕大多數。歷代發展
出\textit{甲骨文、金文、篆書、隸書、草書、楷書}等多種字體。除記錄語言的功能外,漢字的
\emph{藝術性}高。三千餘年歷史中,對日本、韓國、越南諸國語文曾產生極深遠的影響。也稱
為「中國字」、「中國文字」。
\end{document}
